% $Id: problems.tex 109 2006-10-19 21:28:11Z cher $
\documentclass[11pt,oneside]{article}

\usepackage[russian,english]{babel}
\usepackage{cherolymp}
\usepackage{expdlist}
\usepackage{comment}
\usepackage{alltt}
\usepackage{colordvi}
\usepackage{ifthen}

\makeatletter
\def\fancyflag{1}
%\newcommand{\fancyflag}{1}

\begin{document}

\renewcommand{\contestname}{ACM ICPC 2006--2007, NEERC, Moscow Subregional Contest}
\renewcommand{\contestdate}{Moscow, October 21, 2006}

\begin{center}
\LARGE\bf ACM ICPC 2006--2007 NEERC Moscow Subregional Contest\\
Moscow, October 21, 2006
\end{center}

\renewcommand{\tour}{}
\begin{center}
{\Large\textbf{List of Problems}}
\end{center}

\newcommand{\probtableline}[1]{%
\probletter{#1}&\probpage{#1}&\probtime{#1}&\probmem{#1}&\probname{#1}}

{\large
\begin{center}
\begin{tabular}{|c|c|p{2cm}|p{2cm}|p{6.5cm}|}
\hline
Problem & Page & Time limit & Memory limit & Name \\
\hline
\input{probinfo.tex}
\hline
\end{tabular}
\end{center}}

{\large
Your solution must read the input data from the standard input and
write the results to the standard output. Output to the standard
error stream is prohibited.

Unless explicitly stated in the problem statements, all the input
elements may be separated by an arbitrary number of whitespace characters.
All the input data are correct and satisfy specifications given in the
problem statement.

The output of your program must exactly satisfy the output specification
in the problem statement.
}

\begin{center}
\Large\bf General Sponsors
\end{center}

\begin{center}
\includegraphics{rules/cboss.eps}
\includegraphics{rules/yandex.eps}
\end{center}

\newpage

\input{problist.tex}

\end{document}
